%% LyX 2.2.1 created this file.  For more info, see http://www.lyx.org/.
%% Do not edit unless you really know what you are doing.
\documentclass[english,ngerman,AT, texfonts]{dlrreport}
\usepackage[T1]{fontenc}
\usepackage[utf8]{inputenc}
\usepackage{geometry}
\geometry{verbose,tmargin=2.5cm}
\setcounter{secnumdepth}{3}
\setcounter{tocdepth}{3}
\usepackage{color}
\usepackage{refstyle}
\usepackage{float}
\usepackage{textcomp}
\usepackage{amsmath}
\usepackage{amssymb}
\usepackage{graphicx}
\usepackage{nomencl}
% the following is useful when we have the old nomencl.sty package
\providecommand{\printnomenclature}{\printglossary}
\providecommand{\makenomenclature}{\makeglossary}
\makenomenclature

\makeatletter

%%%%%%%%%%%%%%%%%%%%%%%%%%%%%% LyX specific LaTeX commands.

\AtBeginDocument{\providecommand\figref[1]{\ref{fig:#1}}}
%% Because html converters don't know tabularnewline
\providecommand{\tabularnewline}{\\}
\RS@ifundefined{subsecref}
  {\newref{subsec}{name = \RSsectxt}}
  {}
\RS@ifundefined{thmref}
  {\def\RSthmtxt{theorem~}\newref{thm}{name = \RSthmtxt}}
  {}
\RS@ifundefined{lemref}
  {\def\RSlemtxt{lemma~}\newref{lem}{name = \RSlemtxt}}
  {}


%%%%%%%%%%%%%%%%%%%%%%%%%%%%%% User specified LaTeX commands.

\usepackage{nomencl}
\makenomenclature

\usepackage{graphicx}
\usepackage{epstopdf}
%\usepackage{subfigure}

\usepackage{verbatim}
\usepackage{ifthen}
\usepackage{ifpdf} 

\usepackage{lscape}     %% supplies a landscape environment 

% Tabellen
\usepackage{tabularx}
\usepackage{rotating} %benoetigt fuer tabellen quer
\usepackage{booktabs} %Schoenere trennlinien in tabellen


\usepackage{colortbl}
\usepackage{transparent}
\usepackage{framed} 
\usepackage{amsmath}  %spezielle mathematikzeichen wie doppelintegral
\usepackage{amssymb} 

%%%%%%%%%%%%%%%%%%%%%%%%%%%%%%%%%%%%%%%%%%%%%%%%%%%
\usepackage{bibgerm}
%\usepackage{harvard} 
%\renewcommand{\harvardand}{und}
%\usepackage[authoryear]{natbib}
%%%%%%%%%%%%%%%%%%%%%%%%%%%%%%%%%%%%%%%%%%%%%%%%%%%

\usepackage{setspace}
%\onehalfspacing
\spacing{1.2}

%%%%%%%%%%%% Definitionen %%%%%%%%%%%%%%%%%%%%%%%%%

\renewcommand{\familydefault}{\sfdefault}
\renewcommand{\nomname}{Nomenklatur}
\renewcommand{\nomgroup}[1]{%

\ifthenelse{\equal{#1}{L}}{\item[\textbf{Lateinische Symbole}]}{%
\ifthenelse{\equal{#1}{G}}{\item[\textbf{Griechische Symbole}]}{%
\ifthenelse{\equal{#1}{A}}{\item[\textbf{Abkürzungen}]}{%
\ifthenelse{\equal{#1}{S}}{\item[\textbf{Strömungstechnische Bezeichnungen}]}{}}}}}
\def\pagedeclaration#1{\dotfill\nobreakspace#1 }


\makeatletter % @in Befehlen auf code 11 setzen
\newcommand\Kurzfassungname{Kurzfassung}
\newenvironment{Kurzfassung}{%
\titlepage
    \@beginparpenalty\@lowpenalty
     \begin{center}%
       \bfseries \Kurzfassungname
       \@endparpenalty\@M
      \end{center}}%
     {\par\endtitlepage\global\no@Kurzfassungfassungfalse}
\defineasbibitem{\bibitem}
\makeatother % @ wieder auf code 12 zurücksetzen

\definecolor{hellgrau}{gray}{0.8}

%%%%%%%%%%%%%%%%%%%%%%%%%%%%%%%%%%%%%%%%%%%%%%%%%%%


%%%%%%%%%%%% Einstellungen + Extras %%%%%%%%%%%%%%%
\sloppy  % Silbentrennung - keine Trennung bevorzugt auch bei größeren Abständen zwischen den Worten
\makeglossary
%%%%%%%%%%%%%%%%%%%%%%%%%%%%%%%%%%%%%%%%%%%%%%%%%%%



%%%%%%%%%%%%%  Titel, Autor, etc. %%%%%%%%%%%%%%%%%
\title{Multifidelity-Optimierungsverfahren für Turbomaschinen \vspace{0.5cm}  Dissertation  \vspace{0.5cm}  \bf}
\runningtitle{Rachete}
\trname{Dissertation}
\author{Andreas Schmitz}
\date{Oktober 2018}
%%%%%%%%%%%%%%%%%%%%%%%%%%%%%%%%%%%%%%%%%%%%%%%%%%%



%%%%%%%%%%%%%  Schalter %%%%%%%%%%%%%%%%%
\newif\ifcomments
%\commentstrue % comment out to hide comments



%%%%%%%%%%%%%  Titel, Autor, etc. %%%%%%%%%%%%%%%%%
\title{Entwicklung eines Multifidelity Optimierungsverfahrens für hochdimensionale Räume und große Datensätzen \vspace{0.5cm}  Dissertation  }
%%%%%%%%%%%%%%%%%%%%%%%%%%%%%%%%%%%%%%%%%%%%%%%%%%%

\makeatother

\usepackage{babel}
\usepackage{listings}
\addto\captionsenglish{\renewcommand{\lstlistingname}{\inputencoding{latin9}Listing}}
\addto\captionsngerman{\renewcommand{\lstlistingname}{\inputencoding{latin9}Listing}}
\renewcommand{\lstlistingname}{\inputencoding{latin9}Listing}

\begin{document}
\maketitle

\nomenclature{o}{Anzahl der Hyperparameter}\nomenclature{m}{Anzahl der gegebenen partiellen Ableitungen}\nomenclature{$\textrm{var}(X)$}{Varianz der Zufallsvariable X}\nomenclature{$\textrm{cov}(X,Y)$}{Kovarianzfunktion zwischen den Zufallsvariablen X und Y}\nomenclature{$\textrm{cov}(\vec{x},\vec{y})$}{Ortsabhängige Kovarianzfunktion zwischen den Ortsvektoren x und y}\nomenclature{$\mathbf{R}$}{Korrelationsmatrix}\nomenclature{$c(X,Y)$}{Korrelationsfunktion zwischen den Zufallsvariablen X und Y}\nomenclature{$y^{*}\left(\vec{x}\right)$}{Geschätzter Funktionswert an der Stelle $\vec{x}$}\nomenclature{$y\left(\vec{x}\right)$}{Bekannter Funktionswert oder Stützstelle an der Stelle $\vec{x}$}\nomenclature{$w_{i}$}{Gewichte eines Kriging Modells}\nomenclature{$F\left(\vec{x}\right)$}{Fehlerfunktion an der Stelle $\vec{x}$}\nomenclature{$\mathbf{Cov}$}{Kovarianzmatrix}

\nomenclature{$\vec{h}$}{Euklidischer Abstandsvektor}\nomenclature{$h$}{Allgemeiner Hyperparameter }\nomenclature{$\vec{\beta}$}{Beta Vektor, beinhaltet alle Erwartungswerte des Modells.}\nomenclature{${\beta}_{i}$}{Eintrag des Beta Vektors, entspricht einem Erwartungswert. Beim Ordinary Kriging gibt es nur einen Erwartungswert}\nomenclature{$m_{f}$}{Anzahl der verwendenten Fidelities bei Multifidelity Verfahren}

\nomenclature{$\sigma^{2}$}{Varianz} 

\nomenclature{$\vec{r}$}{Korrelationsvektor}

\nomenclature{$\vec{y}_{s}$}{Vektor, welcher alle bekannten Stützstellen enthält}

\nomenclature{$\vec{\theta}$}{Hyperparameter einer Korrelationsfunktion, ohne Varianzen oder Regularisierungsterme}\nomenclature{$L$}{Likelihood Funktion}

\nomenclature{$N$}{Multivariate Normalverteilung}\nomenclature{$\vec{F}$}{Der Vektor entspricht beim Ordinary-Kriging $\vec{1}$ und beim Gradient-Enhanced-Kriging sind die ersten $n-m$ Einträge Eins und für die restlichen $m$ Einträge Null}\nomenclature{$\kappa$}{Konditionszahl}\nomenclature{$\Xi$}{Eigenwert einer Matrix}\nomenclature{$\delta$}{Kronecker Delta}\nomenclature{$\lambda$}{Diagonalaufschlag}\nomenclature{$t$}{Zeit}\nomenclature{$f(...)$}{Entscheidungsfunktion}\nomenclature{$V$}{Kummulierte Volumenzugewinn einer Optimierung}\nomenclature{$P$}{Wahrscheinlichkeit}\nomenclature{$E[X]$}{Erwartungswert der Zufallsvariable X}

{\huge{}Sicher:}{\huge \par}

\textbf{\LARGE{}\nomenclature{SSE}{Streaming SIMD Extensions, eine Befehlssatzerweiterung für Prozessoren.}\nomenclature{AVX}{Advanced Vector Extensions, eine Befehlssatzerweiterung für Prozessoren. Nachfolger der SSE Architektur.}\nomenclature{SIMD}{Single Instruction Multiple Data, eine Architektur welche es erlaubt dieselbe Operation parallel auf einen sich verändernden Datenstrom anzuwenden. }}\nomenclature{k}{Anzahl der freien Variablen eines Members}\nomenclature{s}{Anzahl der verschiedenen Gütestufen bei einem Mutlifidelty Modell}\nomenclature{$\vec{x}$}{Ortsvektor}\nomenclature{$I$}{Informationsgehalt}\nomenclature[A]{Member}{Ein Satz freier Variablen mit dazugehörigen Funktionswerten. Beispielsweise ein Satz geometrischer Parameter mit berechneten mechanischen und aerodynamischen Größen.}\nomenclature{n}{Anzahl der beprobten Stützstellen, auch Member oder Samples genannt}

\textbf{\LARGE{}Noch einbaue n: \cite{LeGratiet}\cite{box1958note}\cite{Eifinger2013}\cite{forrester2009recent}\cite{gibbs1997efficient}\cite{gill1981practical,gill2007numerical}\cite{Sacks2007a}\cite{Jones1998}\cite{keane2006statistical}
\cite{Jones2001}(10dimensions Samples minimum) \cite{Jin2001}(3dimensions
Samples minimum in vielen Fällen)}{\LARGE \par}

\chapter*{{\normalsize{}\Kurzfassung 
\addcontentsline{toc}{chapter}{Kurzfassung}
\abstract 
\addcontentsline{toc}{chapter}{Abstract}}}

\include{Kapitel/Vorwort}

\settowidth{\nomlabelwidth}{$\textrm{cov}(\vec{x},\vec{y})$}
\printnomenclature{}

\tableofcontents{}

\include{Kapitel/Einleitung}

\include{Kapitel/Optimierungsstrategie}\include{Kapitel/OptimierungsstrategieErweiterung}

\include{Kapitel/Kriging}\include{Kapitel/SoftwaretechnischeUmsetzung}

\include{Kapitel/Benchmarks}

\include{Kapitel/RealeTurbomaschinenOptimierung}

\include{Kapitel/AusblickFazit}

\include{Kapitel/Anhang}

\bibliographystyle{ieeetr}
\bibliography{library}

\end{document}
